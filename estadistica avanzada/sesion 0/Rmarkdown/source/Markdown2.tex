% Options for packages loaded elsewhere
\PassOptionsToPackage{unicode}{hyperref}
\PassOptionsToPackage{hyphens}{url}
%
\documentclass[
]{article}
\usepackage{amsmath,amssymb}
\usepackage{lmodern}
\usepackage{iftex}
\ifPDFTeX
  \usepackage[T1]{fontenc}
  \usepackage[utf8]{inputenc}
  \usepackage{textcomp} % provide euro and other symbols
\else % if luatex or xetex
  \usepackage{unicode-math}
  \defaultfontfeatures{Scale=MatchLowercase}
  \defaultfontfeatures[\rmfamily]{Ligatures=TeX,Scale=1}
\fi
% Use upquote if available, for straight quotes in verbatim environments
\IfFileExists{upquote.sty}{\usepackage{upquote}}{}
\IfFileExists{microtype.sty}{% use microtype if available
  \usepackage[]{microtype}
  \UseMicrotypeSet[protrusion]{basicmath} % disable protrusion for tt fonts
}{}
\makeatletter
\@ifundefined{KOMAClassName}{% if non-KOMA class
  \IfFileExists{parskip.sty}{%
    \usepackage{parskip}
  }{% else
    \setlength{\parindent}{0pt}
    \setlength{\parskip}{6pt plus 2pt minus 1pt}}
}{% if KOMA class
  \KOMAoptions{parskip=half}}
\makeatother
\usepackage{xcolor}
\IfFileExists{xurl.sty}{\usepackage{xurl}}{} % add URL line breaks if available
\IfFileExists{bookmark.sty}{\usepackage{bookmark}}{\usepackage{hyperref}}
\hypersetup{
  pdftitle={Managing and Understanding Data},
  pdfauthor={Escribir vuestro nombre y apellidos},
  hidelinks,
  pdfcreator={LaTeX via pandoc}}
\urlstyle{same} % disable monospaced font for URLs
\usepackage[margin=1in]{geometry}
\usepackage{color}
\usepackage{fancyvrb}
\newcommand{\VerbBar}{|}
\newcommand{\VERB}{\Verb[commandchars=\\\{\}]}
\DefineVerbatimEnvironment{Highlighting}{Verbatim}{commandchars=\\\{\}}
% Add ',fontsize=\small' for more characters per line
\usepackage{framed}
\definecolor{shadecolor}{RGB}{248,248,248}
\newenvironment{Shaded}{\begin{snugshade}}{\end{snugshade}}
\newcommand{\AlertTok}[1]{\textcolor[rgb]{0.94,0.16,0.16}{#1}}
\newcommand{\AnnotationTok}[1]{\textcolor[rgb]{0.56,0.35,0.01}{\textbf{\textit{#1}}}}
\newcommand{\AttributeTok}[1]{\textcolor[rgb]{0.77,0.63,0.00}{#1}}
\newcommand{\BaseNTok}[1]{\textcolor[rgb]{0.00,0.00,0.81}{#1}}
\newcommand{\BuiltInTok}[1]{#1}
\newcommand{\CharTok}[1]{\textcolor[rgb]{0.31,0.60,0.02}{#1}}
\newcommand{\CommentTok}[1]{\textcolor[rgb]{0.56,0.35,0.01}{\textit{#1}}}
\newcommand{\CommentVarTok}[1]{\textcolor[rgb]{0.56,0.35,0.01}{\textbf{\textit{#1}}}}
\newcommand{\ConstantTok}[1]{\textcolor[rgb]{0.00,0.00,0.00}{#1}}
\newcommand{\ControlFlowTok}[1]{\textcolor[rgb]{0.13,0.29,0.53}{\textbf{#1}}}
\newcommand{\DataTypeTok}[1]{\textcolor[rgb]{0.13,0.29,0.53}{#1}}
\newcommand{\DecValTok}[1]{\textcolor[rgb]{0.00,0.00,0.81}{#1}}
\newcommand{\DocumentationTok}[1]{\textcolor[rgb]{0.56,0.35,0.01}{\textbf{\textit{#1}}}}
\newcommand{\ErrorTok}[1]{\textcolor[rgb]{0.64,0.00,0.00}{\textbf{#1}}}
\newcommand{\ExtensionTok}[1]{#1}
\newcommand{\FloatTok}[1]{\textcolor[rgb]{0.00,0.00,0.81}{#1}}
\newcommand{\FunctionTok}[1]{\textcolor[rgb]{0.00,0.00,0.00}{#1}}
\newcommand{\ImportTok}[1]{#1}
\newcommand{\InformationTok}[1]{\textcolor[rgb]{0.56,0.35,0.01}{\textbf{\textit{#1}}}}
\newcommand{\KeywordTok}[1]{\textcolor[rgb]{0.13,0.29,0.53}{\textbf{#1}}}
\newcommand{\NormalTok}[1]{#1}
\newcommand{\OperatorTok}[1]{\textcolor[rgb]{0.81,0.36,0.00}{\textbf{#1}}}
\newcommand{\OtherTok}[1]{\textcolor[rgb]{0.56,0.35,0.01}{#1}}
\newcommand{\PreprocessorTok}[1]{\textcolor[rgb]{0.56,0.35,0.01}{\textit{#1}}}
\newcommand{\RegionMarkerTok}[1]{#1}
\newcommand{\SpecialCharTok}[1]{\textcolor[rgb]{0.00,0.00,0.00}{#1}}
\newcommand{\SpecialStringTok}[1]{\textcolor[rgb]{0.31,0.60,0.02}{#1}}
\newcommand{\StringTok}[1]{\textcolor[rgb]{0.31,0.60,0.02}{#1}}
\newcommand{\VariableTok}[1]{\textcolor[rgb]{0.00,0.00,0.00}{#1}}
\newcommand{\VerbatimStringTok}[1]{\textcolor[rgb]{0.31,0.60,0.02}{#1}}
\newcommand{\WarningTok}[1]{\textcolor[rgb]{0.56,0.35,0.01}{\textbf{\textit{#1}}}}
\usepackage{longtable,booktabs,array}
\usepackage{calc} % for calculating minipage widths
% Correct order of tables after \paragraph or \subparagraph
\usepackage{etoolbox}
\makeatletter
\patchcmd\longtable{\par}{\if@noskipsec\mbox{}\fi\par}{}{}
\makeatother
% Allow footnotes in longtable head/foot
\IfFileExists{footnotehyper.sty}{\usepackage{footnotehyper}}{\usepackage{footnote}}
\makesavenoteenv{longtable}
\usepackage{graphicx}
\makeatletter
\def\maxwidth{\ifdim\Gin@nat@width>\linewidth\linewidth\else\Gin@nat@width\fi}
\def\maxheight{\ifdim\Gin@nat@height>\textheight\textheight\else\Gin@nat@height\fi}
\makeatother
% Scale images if necessary, so that they will not overflow the page
% margins by default, and it is still possible to overwrite the defaults
% using explicit options in \includegraphics[width, height, ...]{}
\setkeys{Gin}{width=\maxwidth,height=\maxheight,keepaspectratio}
% Set default figure placement to htbp
\makeatletter
\def\fps@figure{htbp}
\makeatother
\setlength{\emergencystretch}{3em} % prevent overfull lines
\providecommand{\tightlist}{%
  \setlength{\itemsep}{0pt}\setlength{\parskip}{0pt}}
\setcounter{secnumdepth}{-\maxdimen} % remove section numbering
\ifLuaTeX
  \usepackage{selnolig}  % disable illegal ligatures
\fi
\newlength{\cslhangindent}
\setlength{\cslhangindent}{1.5em}
\newlength{\csllabelwidth}
\setlength{\csllabelwidth}{3em}
\newenvironment{CSLReferences}[2] % #1 hanging-ident, #2 entry spacing
 {% don't indent paragraphs
  \setlength{\parindent}{0pt}
  % turn on hanging indent if param 1 is 1
  \ifodd #1 \everypar{\setlength{\hangindent}{\cslhangindent}}\ignorespaces\fi
  % set entry spacing
  \ifnum #2 > 0
  \setlength{\parskip}{#2\baselineskip}
  \fi
 }%
 {}
\usepackage{calc}
\newcommand{\CSLBlock}[1]{#1\hfill\break}
\newcommand{\CSLLeftMargin}[1]{\parbox[t]{\csllabelwidth}{#1}}
\newcommand{\CSLRightInline}[1]{\parbox[t]{\linewidth - \csllabelwidth}{#1}\break}
\newcommand{\CSLIndent}[1]{\hspace{\cslhangindent}#1}

\title{Managing and Understanding Data}
\author{Escribir vuestro nombre y apellidos}
\date{22 de febrero, 2022}

\begin{document}
\maketitle

{
\setcounter{tocdepth}{2}
\tableofcontents
}
\hypertarget{r-data-structures}{%
\section{R data structures}\label{r-data-structures}}

The R data structures used most frequently in machine learning are
\emph{vectors}, \emph{factors}, \emph{lists}, \emph{arrays}, and
\emph{data frames}.

To find out more about statistical methods with R see (Teetor 2011;
Hothorn and Everitt 2014; Baayen 2008).

\hypertarget{vectors}{%
\subsection{Vectors}\label{vectors}}

The fundamental R data structure is the \textbf{vector}, which stores an
ordered set of values called \textbf{elements}. A vector can contain any
number of elements. However, all the elements must be of the same type;
for instance, a vector cannot contain both numbers and text.

There are several vector types commonly used in machine
learning:\texttt{integer}(numbers without decimals), \texttt{numeric}
(numbers with decimals), \texttt{character} (text data), or
\texttt{logical} (\texttt{TRUE} or \texttt{FALSE} values). There are
also two special values: \texttt{NULL}, which is used to indicate the
absence of any value, and \texttt{NA}, which indicates a missing value.

Create vectors of data for three medical patients:

\begin{Shaded}
\begin{Highlighting}[]
\CommentTok{\# create vectors of data for three medical patients}
\NormalTok{subject\_name }\OtherTok{\textless{}{-}} \FunctionTok{c}\NormalTok{(}\StringTok{"John Doe"}\NormalTok{, }\StringTok{"Jane Doe"}\NormalTok{, }\StringTok{"Steve Graves"}\NormalTok{)}
\NormalTok{temperature }\OtherTok{\textless{}{-}} \FunctionTok{c}\NormalTok{(}\FloatTok{98.1}\NormalTok{, }\FloatTok{98.6}\NormalTok{, }\FloatTok{101.4}\NormalTok{)}
\NormalTok{flu\_status }\OtherTok{\textless{}{-}} \FunctionTok{c}\NormalTok{(}\ConstantTok{FALSE}\NormalTok{, }\ConstantTok{FALSE}\NormalTok{, }\ConstantTok{TRUE}\NormalTok{)}
\end{Highlighting}
\end{Shaded}

Access the second element in body temperature vector:

\begin{Shaded}
\begin{Highlighting}[]
\CommentTok{\# access the second element in body temperature vector}
\NormalTok{temperature[}\DecValTok{2}\NormalTok{]}
\end{Highlighting}
\end{Shaded}

\begin{verbatim}
## [1] 98.6
\end{verbatim}

Examples of accessing items in vector include items in the range 2 to 3.

\begin{Shaded}
\begin{Highlighting}[]
\DocumentationTok{\#\# examples of accessing items in vector}
\CommentTok{\# include items in the range 2 to 3}
\NormalTok{temperature[}\DecValTok{2}\SpecialCharTok{:}\DecValTok{3}\NormalTok{]}
\end{Highlighting}
\end{Shaded}

\begin{verbatim}
## [1]  98.6 101.4
\end{verbatim}

Exclude item 2 using the minus sign

\begin{Shaded}
\begin{Highlighting}[]
\CommentTok{\# exclude item 2 using the minus sign}
\NormalTok{temperature[}\SpecialCharTok{{-}}\DecValTok{2}\NormalTok{]}
\end{Highlighting}
\end{Shaded}

\begin{verbatim}
## [1]  98.1 101.4
\end{verbatim}

Use a vector to indicate whether to include item

\begin{Shaded}
\begin{Highlighting}[]
\CommentTok{\# use a vector to indicate whether to include item}
\NormalTok{temperature[}\FunctionTok{c}\NormalTok{(}\ConstantTok{TRUE}\NormalTok{, }\ConstantTok{TRUE}\NormalTok{, }\ConstantTok{FALSE}\NormalTok{)]}
\end{Highlighting}
\end{Shaded}

\begin{verbatim}
## [1] 98.1 98.6
\end{verbatim}

\hypertarget{exploring-and-understanding-data}{%
\section{Exploring and understanding
data}\label{exploring-and-understanding-data}}

After collecting data and loading it into R data structures, the next
step in the machine learning process involves examining the data in
detail. It is during this step that you will begin to explore the data's
features and examples, and realize the peculiarities that make your data
unique. The better you understand your data, the better you will be able
to match a machine learning model to your learning problem. The best way
to understand the process of data exploration is by example. In this
section, we will explore the \texttt{usedcars.csv} dataset, which
contains actual data about used cars recently advertised for sale on a
popular U.S. website.

Since the dataset is stored in CSV form, we can use the
\texttt{read.csv()} function to load the data into an R data frame:

\begin{Shaded}
\begin{Highlighting}[]
\DocumentationTok{\#\#\#\#\# Exploring and understanding data {-}{-}{-}{-}{-}{-}{-}{-}{-}{-}{-}{-}{-}{-}{-}{-}{-}{-}{-}{-}}

\DocumentationTok{\#\# data exploration example using used car data}
\NormalTok{usedcars }\OtherTok{\textless{}{-}} \FunctionTok{read.csv}\NormalTok{(}\StringTok{"usedcars.csv"}\NormalTok{, }\AttributeTok{stringsAsFactors =} \ConstantTok{FALSE}\NormalTok{)}
\end{Highlighting}
\end{Shaded}

\hypertarget{exploring-the-structure-of-data}{%
\subsection{Exploring the structure of
data}\label{exploring-the-structure-of-data}}

One of the first questions to ask in your investigation should be about
how data is organized. If you are fortunate, your source will provide a
\textbf{data dictionary}, a document that describes the data's features.
In our case, the used car data does not come with this documentation, so
we'll need to create our own.

\begin{Shaded}
\begin{Highlighting}[]
\CommentTok{\# get structure of used car data}
\FunctionTok{str}\NormalTok{(usedcars)}
\end{Highlighting}
\end{Shaded}

\begin{verbatim}
## 'data.frame':    150 obs. of  6 variables:
##  $ year        : int  2011 2011 2011 2011 2012 2010 2011 2010 2011 2010 ...
##  $ model       : chr  "SEL" "SEL" "SEL" "SEL" ...
##  $ price       : int  21992 20995 19995 17809 17500 17495 17000 16995 16995 16995 ...
##  $ mileage     : int  7413 10926 7351 11613 8367 25125 27393 21026 32655 36116 ...
##  $ color       : chr  "Yellow" "Gray" "Silver" "Gray" ...
##  $ transmission: chr  "AUTO" "AUTO" "AUTO" "AUTO" ...
\end{verbatim}

\hypertarget{exploring-numeric-variables}{%
\subsection{Exploring numeric
variables}\label{exploring-numeric-variables}}

\begin{Shaded}
\begin{Highlighting}[]
\DocumentationTok{\#\# Exploring numeric variables {-}{-}{-}{-}{-}}

\CommentTok{\# summarize numeric variables}
\FunctionTok{summary}\NormalTok{(usedcars}\SpecialCharTok{$}\NormalTok{year)}
\end{Highlighting}
\end{Shaded}

\begin{verbatim}
##    Min. 1st Qu.  Median    Mean 3rd Qu.    Max. 
##    2000    2008    2009    2009    2010    2012
\end{verbatim}

\begin{Shaded}
\begin{Highlighting}[]
\FunctionTok{summary}\NormalTok{(usedcars[}\FunctionTok{c}\NormalTok{(}\StringTok{"price"}\NormalTok{, }\StringTok{"mileage"}\NormalTok{)])}
\end{Highlighting}
\end{Shaded}

\begin{verbatim}
##      price          mileage      
##  Min.   : 3800   Min.   :  4867  
##  1st Qu.:10995   1st Qu.: 27200  
##  Median :13592   Median : 36385  
##  Mean   :12962   Mean   : 44261  
##  3rd Qu.:14904   3rd Qu.: 55125  
##  Max.   :21992   Max.   :151479
\end{verbatim}

\begin{Shaded}
\begin{Highlighting}[]
\CommentTok{\# calculate the mean income}
\NormalTok{(}\DecValTok{36000} \SpecialCharTok{+} \DecValTok{44000} \SpecialCharTok{+} \DecValTok{56000}\NormalTok{) }\SpecialCharTok{/} \DecValTok{3}
\end{Highlighting}
\end{Shaded}

\begin{verbatim}
## [1] 45333.33
\end{verbatim}

\begin{Shaded}
\begin{Highlighting}[]
\FunctionTok{mean}\NormalTok{(}\FunctionTok{c}\NormalTok{(}\DecValTok{36000}\NormalTok{, }\DecValTok{44000}\NormalTok{, }\DecValTok{56000}\NormalTok{))}
\end{Highlighting}
\end{Shaded}

\begin{verbatim}
## [1] 45333.33
\end{verbatim}

\begin{Shaded}
\begin{Highlighting}[]
\CommentTok{\# the median income}
\FunctionTok{median}\NormalTok{(}\FunctionTok{c}\NormalTok{(}\DecValTok{36000}\NormalTok{, }\DecValTok{44000}\NormalTok{, }\DecValTok{56000}\NormalTok{))}
\end{Highlighting}
\end{Shaded}

\begin{verbatim}
## [1] 44000
\end{verbatim}

\begin{Shaded}
\begin{Highlighting}[]
\CommentTok{\# the min/max of used car prices}
\FunctionTok{range}\NormalTok{(usedcars}\SpecialCharTok{$}\NormalTok{price)}
\end{Highlighting}
\end{Shaded}

\begin{verbatim}
## [1]  3800 21992
\end{verbatim}

\begin{Shaded}
\begin{Highlighting}[]
\CommentTok{\# the difference of the range}
\FunctionTok{diff}\NormalTok{(}\FunctionTok{range}\NormalTok{(usedcars}\SpecialCharTok{$}\NormalTok{price))}
\end{Highlighting}
\end{Shaded}

\begin{verbatim}
## [1] 18192
\end{verbatim}

\begin{Shaded}
\begin{Highlighting}[]
\CommentTok{\# IQR for used car prices}
\FunctionTok{IQR}\NormalTok{(usedcars}\SpecialCharTok{$}\NormalTok{price)}
\end{Highlighting}
\end{Shaded}

\begin{verbatim}
## [1] 3909.5
\end{verbatim}

\begin{Shaded}
\begin{Highlighting}[]
\CommentTok{\# use quantile to calculate five{-}number summary}
\FunctionTok{quantile}\NormalTok{(usedcars}\SpecialCharTok{$}\NormalTok{price)}
\end{Highlighting}
\end{Shaded}

\begin{verbatim}
##      0%     25%     50%     75%    100% 
##  3800.0 10995.0 13591.5 14904.5 21992.0
\end{verbatim}

\begin{Shaded}
\begin{Highlighting}[]
\CommentTok{\# the 99th percentile}
\FunctionTok{quantile}\NormalTok{(usedcars}\SpecialCharTok{$}\NormalTok{price, }\AttributeTok{probs =} \FunctionTok{c}\NormalTok{(}\FloatTok{0.01}\NormalTok{, }\FloatTok{0.99}\NormalTok{))}
\end{Highlighting}
\end{Shaded}

\begin{verbatim}
##       1%      99% 
##  5428.69 20505.00
\end{verbatim}

\begin{Shaded}
\begin{Highlighting}[]
\CommentTok{\# quintiles}
\FunctionTok{quantile}\NormalTok{(usedcars}\SpecialCharTok{$}\NormalTok{price, }\FunctionTok{seq}\NormalTok{(}\AttributeTok{from =} \DecValTok{0}\NormalTok{, }\AttributeTok{to =} \DecValTok{1}\NormalTok{, }\AttributeTok{by =} \FloatTok{0.20}\NormalTok{))}
\end{Highlighting}
\end{Shaded}

\begin{verbatim}
##      0%     20%     40%     60%     80%    100% 
##  3800.0 10759.4 12993.8 13992.0 14999.0 21992.0
\end{verbatim}

\hypertarget{visualizing-numeric-variables---boxplots}{%
\subsection{Visualizing numeric variables -
boxplots}\label{visualizing-numeric-variables---boxplots}}

\begin{Shaded}
\begin{Highlighting}[]
\CommentTok{\# boxplot of used car prices and mileage}
\FunctionTok{boxplot}\NormalTok{(usedcars}\SpecialCharTok{$}\NormalTok{price, }\AttributeTok{main=}\StringTok{"Boxplot of Used Car Prices"}\NormalTok{,}\AttributeTok{ylab=}\StringTok{"Price ($)"}\NormalTok{)}
\end{Highlighting}
\end{Shaded}



\begin{Shaded}
\begin{Highlighting}[]
\FunctionTok{boxplot}\NormalTok{(usedcars}\SpecialCharTok{$}\NormalTok{mileage, }\AttributeTok{main=}\StringTok{"Boxplot of Used Car Mileage"}\NormalTok{,}
      \AttributeTok{ylab=}\StringTok{"Odometer (mi.)"}\NormalTok{)}
\end{Highlighting}
\end{Shaded}

\includegraphics{Markdown_files/figure-latex/graphics-2.pdf}

\begin{Shaded}
\begin{Highlighting}[]
\CommentTok{\# histograms of used car prices and mileage}
\FunctionTok{hist}\NormalTok{(usedcars}\SpecialCharTok{$}\NormalTok{price, }\AttributeTok{main =} \StringTok{"Histogram of Used Car Prices"}\NormalTok{,}
     \AttributeTok{xlab =} \StringTok{"Price ($)"}\NormalTok{)}
\end{Highlighting}
\end{Shaded}

\includegraphics{Markdown_files/figure-latex/graphics-3.pdf}

\begin{Shaded}
\begin{Highlighting}[]
\FunctionTok{hist}\NormalTok{(usedcars}\SpecialCharTok{$}\NormalTok{mileage, }\AttributeTok{main =} \StringTok{"Histogram of Used Car Mileage"}\NormalTok{,}
     \AttributeTok{xlab =} \StringTok{"Odometer (mi.)"}\NormalTok{)}
\end{Highlighting}
\end{Shaded}

\includegraphics{Markdown_files/figure-latex/graphics-4.pdf}

\begin{Shaded}
\begin{Highlighting}[]
\CommentTok{\# variance and standard deviation of the used car data}
\FunctionTok{var}\NormalTok{(usedcars}\SpecialCharTok{$}\NormalTok{price)}
\end{Highlighting}
\end{Shaded}

\begin{verbatim}
## [1] 9749892
\end{verbatim}

\begin{Shaded}
\begin{Highlighting}[]
\FunctionTok{sd}\NormalTok{(usedcars}\SpecialCharTok{$}\NormalTok{price)}
\end{Highlighting}
\end{Shaded}

\begin{verbatim}
## [1] 3122.482
\end{verbatim}

\begin{Shaded}
\begin{Highlighting}[]
\FunctionTok{var}\NormalTok{(usedcars}\SpecialCharTok{$}\NormalTok{mileage)}
\end{Highlighting}
\end{Shaded}

\begin{verbatim}
## [1] 728033954
\end{verbatim}

\begin{Shaded}
\begin{Highlighting}[]
\FunctionTok{sd}\NormalTok{(usedcars}\SpecialCharTok{$}\NormalTok{mileage)}
\end{Highlighting}
\end{Shaded}

\begin{verbatim}
## [1] 26982.1
\end{verbatim}

\hypertarget{measuring-spread---quartiles-and-the-five-number-summary}{%
\subsection{Measuring spread - quartiles and the five-number
summary}\label{measuring-spread---quartiles-and-the-five-number-summary}}

The \textbf{five-number summary} is a set of five statistics that
roughly depict the spread of a dataset. All five of the statistics are
included in the output of the \texttt{summary()} function. Written in
order, they are:

\begin{enumerate}
\def\labelenumi{\arabic{enumi}.}
\tightlist
\item
  Minimum (\texttt{Min.})
\item
  First quartile, or Q1 (\texttt{1st\ Qu.})
\item
  Median, or Q2 (\texttt{Median})
\item
  Third quartile, or Q3 (\texttt{3rd\ Qu.})
\item
  Maximum (\texttt{Max.})
\end{enumerate}

\hypertarget{measuring-spread---variance-and-standard-deviation}{%
\subsection{Measuring spread - variance and standard
deviation}\label{measuring-spread---variance-and-standard-deviation}}

In order to calculate the standard deviation, we must first obtain the
\textbf{variance}, which is defined as the average of the squared
differences between each value and the mean value. In mathematical
notation, the variance of a set of \texttt{n} values of \texttt{x} is
defined by the following formula. The Greek letter \texttt{mu} (\(\mu\))
(similar in appearance to an m) denotes the mean of the values, and the
variance itself is denoted by the Greek letter \texttt{sigma}
(\(\sigma\)) squared (similar to a b turned sideways):

\[
Var(X)= \sigma^2 = \frac{1}{n}\sum_{i=1}^n (x_i - \mu)^2
\]

The standard deviation is the square root of the variance, and is
denoted by \texttt{sigma} as shown in the following formula:

\[
StdDev(X)= \sigma = \sqrt{\frac{1}{n}\sum_{i=1}^n (x_i - \mu)^2}
\]

\emph{Note}. For more details on using mathematical expressions in Latex
(R Markdown) see
\url{https://es.sharelatex.com/learn/Mathematical_expressions}.

\#\#Table with information about mileage and price

\begin{Shaded}
\begin{Highlighting}[]
\NormalTok{mileage\_s}\OtherTok{\textless{}{-}}\FunctionTok{summary}\NormalTok{(usedcars}\SpecialCharTok{$}\NormalTok{mileage)}
\NormalTok{price\_s}\OtherTok{\textless{}{-}}\FunctionTok{summary}\NormalTok{(usedcars}\SpecialCharTok{$}\NormalTok{price)}
\FunctionTok{kable}\NormalTok{(}\FunctionTok{rbind}\NormalTok{(mileage\_s,price\_s))}
\end{Highlighting}
\end{Shaded}

\begin{longtable}[]{@{}lrrrrrr@{}}
\toprule
& Min. & 1st Qu. & Median & Mean & 3rd Qu. & Max. \\
\midrule
\endhead
mileage\_s & 4867 & 27200.25 & 36385.0 & 44260.65 & 55124.5 & 151479 \\
price\_s & 3800 & 10995.00 & 13591.5 & 12961.93 & 14904.5 & 21992 \\
\bottomrule
\end{longtable}

\#\#Some descriptive graphics

\begin{Shaded}
\begin{Highlighting}[]
\FunctionTok{par}\NormalTok{(}\AttributeTok{mfrow=}\FunctionTok{c}\NormalTok{(}\DecValTok{2}\NormalTok{,}\DecValTok{2}\NormalTok{))}
\FunctionTok{hist}\NormalTok{(usedcars}\SpecialCharTok{$}\NormalTok{mileage,}\AttributeTok{xlab=}\StringTok{"Mileage"}\NormalTok{,}\AttributeTok{main=}\StringTok{"Histogram of mileage"}\NormalTok{,}\AttributeTok{col=}\StringTok{"grey85"}\NormalTok{)}
\FunctionTok{hist}\NormalTok{(usedcars}\SpecialCharTok{$}\NormalTok{price,}\AttributeTok{xlab=}\StringTok{"Price"}\NormalTok{,}\AttributeTok{main=}\StringTok{"Histogram of price"}\NormalTok{,}\AttributeTok{col=}\StringTok{"grey85"}\NormalTok{)}
\NormalTok{usedcars}\SpecialCharTok{$}\NormalTok{transmission}\OtherTok{\textless{}{-}}\FunctionTok{factor}\NormalTok{(usedcars}\SpecialCharTok{$}\NormalTok{transmission)}
\FunctionTok{plot}\NormalTok{(usedcars}\SpecialCharTok{$}\NormalTok{mileage,usedcars}\SpecialCharTok{$}\NormalTok{price,}\AttributeTok{pch=}\DecValTok{16}\NormalTok{,}\AttributeTok{col=}\NormalTok{usedcars}\SpecialCharTok{$}\NormalTok{transmission,}\AttributeTok{xlab=}\StringTok{"Mileage"}\NormalTok{,}\AttributeTok{ylab=}\StringTok{"Price"}\NormalTok{)}
\FunctionTok{legend}\NormalTok{(}\StringTok{"topright"}\NormalTok{,}\AttributeTok{pch=}\DecValTok{16}\NormalTok{,}\FunctionTok{c}\NormalTok{(}\StringTok{"AUTO"}\NormalTok{,}\StringTok{"MANUAL"}\NormalTok{),}\AttributeTok{col=}\DecValTok{1}\SpecialCharTok{:}\DecValTok{2}\NormalTok{,}\AttributeTok{cex=}\FloatTok{0.5}\NormalTok{)}
\end{Highlighting}
\end{Shaded}

\includegraphics{Markdown_files/figure-latex/unnamed-chunk-2-1.pdf}

\#References

\hypertarget{refs}{}
\begin{CSLReferences}{1}{0}
\leavevmode\vadjust pre{\hypertarget{ref-baayen2008analyzing}{}}%
Baayen, R Harald. 2008. \emph{Analyzing Linguistic Data: A Practical
Introduction to Statistics Using r}. Cambridge University Press.

\leavevmode\vadjust pre{\hypertarget{ref-hothorn2014handbook}{}}%
Hothorn, Torsten, and Brian S Everitt. 2014. \emph{A Handbook of
Statistical Analyses Using r}. CRC press.

\leavevmode\vadjust pre{\hypertarget{ref-teetor2011r}{}}%
Teetor, Paul. 2011. \emph{R Cookbook: Proven Recipes for Data Analysis,
Statistics, and Graphics}. " O'Reilly Media, Inc.".

\end{CSLReferences}

\end{document}
